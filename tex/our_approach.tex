\section{Our Approach}\label{our-approach}
%\label{sec:associativity}
Fundamentally, all prior techniques present the following maxim -- one
can either implement large power-hungry TLBs to accommodate all the
benefits of fully-flexible VM, or one can get away with smaller TLBs
as long as software flexibility is sacrificed. In the first scenario,
large TLBs are essential because VM is fully associative. In the
second scenario, software flexibility is sacrificed because VM imposes
restrictions reminiscent of direct mapping. SpryVM's contribution is
to showcase, however, that there is a third regime of operation
between these two extremes, where we implement set-associative VM to
continue enjoying the benefits of flexible VM without requiring large
TLBs. 

The core idea behind SpryVM is (partly) captured in Figure 1. With
SpryVM, VM associativity is restricted in that a virtual address is
guaranteed to map to a physical address in a specific memory
``region''. This memory region could be a socket or a memory
channel. This effective set-associativity provides two
benefits. First, it enables us to replace large per-CPU/accelerator
TLB that need to address the entire physical address space with
memory-side TLBs responsible for mapping only their specific memory
regions. Because memory regions represent just a subset of the total
physical address space, memory-side TLBs can achieve high hit rates
despite being smaller in size. Second, set-associativity also permits
us to reduce TLB miss by allowing us to co-locate data and
translations in the same memory regions, partly overlapping their
lookup. The confluence of these two benefits is a combination of
better performance and reduction in TLB resources. 

A factor in our approach is that implementing set-associativity in a
commodity OS requires only modest changes to existing memory
allocation/management code-paths. Furthermore, we have found that in
practice, we can achieve good address translation performance by only
modestly reducing associativity from the default fully-associative VM
approach. Consequently, SpryVM implements unified CPU-accelerator
address spaces without forcing coarser-grained memory protection,
compromising on entropy in virtual address bits, or requiring
power/area-hungry TLBs. The fact that SpryVM does not rely on
contiguity also means that it can continue to operate efficiently in
fragmented memory settings where translation contiguity or identity
mappings may be hard to generate. Furthemore, the only modest drop
from fully-associative designs means that optimizations like
copy-on-write can, for all practical purposes, be supported (we will
show that the number of CoW pages is limited only by the size of the
memory region, which can be hundreds/thousands of pages in practice).

