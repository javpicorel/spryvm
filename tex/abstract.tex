\begin{abstract}

As the computing industry embraces heterogeneity, a key research
challenge is the question of how to integrate hardware accelerators in
modern platforms without impacting the conventional virtual memory
abstractions modern complex software stacks rely on. Unfortunately, VM
support for accelerators is fundamentally different from CPU's due to
prohibitive TLB reach requirements, long-latency page table walks and
tight area and power budgets. Moreover, recent proposals for accelerator 
VM support break conventional VM
abstractions with intrusive solutions to software stack to facilitate
address translation.

We propose spryVM, a set-associate VM for in-memory workloads. We show
that maintaining the portion of the address space accelerators operate
on set-associatively has little impact on page fault traffic, but
reduces address translation hardware requirements. SpryVM leverages
this observation to improve TLB hit rates by desigining TLBs targeted
to accelerator-local partitions, and by breaking page table walk-data
fetch serialization upon TLB misses. SpryVM achieves within 1.2\% and
0.6\% of ideal translation in scenarios where working sets are
memory-resident and exceed the available memory capacity,
respectively. Finally, we implement spryVM in stock Linux, showing
that these benefits are achievable while reatining conventional VM
abstractions and with only modest changes to existing VM software
stacks.



\end{abstract}

